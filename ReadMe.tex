% !TEX TS-program = pdflatex
% !TEX encoding = UTF-8 Unicode

% This is a simple template for a LaTeX document using the "article" class.
% See "book", "report", "letter" for other types of document.

\documentclass[11pt]{article} % use larger type; default would be 10pt

\usepackage[utf8]{inputenc} % set input encoding (not needed with XeLaTeX)

%%% Examples of Article customizations
% These packages are optional, depending whether you want the features they provide.
% See the LaTeX Companion or other references for full information.

%%% PAGE DIMENSIONS
\usepackage{geometry} % to change the page dimensions
\geometry{a4paper} % or letterpaper (US) or a5paper or....
% \geometry{margin=2in} % for example, change the margins to 2 inches all round
% \geometry{landscape} % set up the page for landscape
%   read geometry.pdf for detailed page layout information

\usepackage{graphicx} % support the \includegraphics command and options

% \usepackage[parfill]{parskip} % Activate to begin paragraphs with an empty line rather than an indent

%%% PACKAGES
\usepackage{booktabs} % for much better looking tables
\usepackage{array} % for better arrays (eg matrices) in maths
\usepackage{paralist} % very flexible & customisable lists (eg. enumerate/itemize, etc.)
\usepackage{verbatim} % adds environment for commenting out blocks of text & for better verbatim
\usepackage{subfig} % make it possible to include more than one captioned figure/table in a single float
\usepackage{amsmath, amsfonts}
% These packages are all incorporated in the memoir class to one degree or another...

%%% HEADERS & FOOTERS
\usepackage{fancyhdr} % This should be set AFTER setting up the page geometry
\pagestyle{fancy} % options: empty , plain , fancy
\renewcommand{\headrulewidth}{0pt} % customise the layout...
\lhead{}\chead{}\rhead{}
\lfoot{}\cfoot{\thepage}\rfoot{}

%%% SECTION TITLE APPEARANCE
\usepackage{sectsty}
\allsectionsfont{\sffamily\mdseries\upshape} % (See the fntguide.pdf for font help)
% (This matches ConTeXt defaults)

%%% ToC (table of contents) APPEARANCE
\usepackage[nottoc,notlof,notlot]{tocbibind} % Put the bibliography in the ToC
\usepackage[titles,subfigure]{tocloft} % Alter the style of the Table of Contents
\renewcommand{\cftsecfont}{\rmfamily\mdseries\upshape}
\renewcommand{\cftsecpagefont}{\rmfamily\mdseries\upshape} % No bold!

%%% END Article customizations

%%% The "real" document content comes below...

\title{Conjugacy tables with fixed margins}
\date{} % Activate to display a given date or no date (if empty),
         % otherwise the current date is printed 

\begin{document}
\maketitle

\section{Background}

Let $\lambda = (\lambda_1, \ldots, \lambda_m)$  and $\mu = (\mu_1, \ldots, \mu_n)$ be compositions of a natural number $d$. Let $M$ be an $m \times n$ matrix with entries in $\mathbb{N}$. Say that the pair $(\lambda, \mu)$ is the {\em margin} of $M$ if the sum of entries in the $i$-th row of $M$ is $\lambda_i$, and the sum of entries in the $j$-th column of $M$ is $\mu_j$.
Write $A^{\lambda}_{\mu}$ for the set of such matrices. 

For example:
\begin{align*}
A^{(3,3)}_{(2,2,2)} 
= \{
&
\begin{pmatrix}
    2 & 1 & 0 \\
    0 & 1 & 2\\
  \end{pmatrix}
,
\begin{pmatrix}
    2 & 0 & 1 \\
    0 & 2 & 1\\
  \end{pmatrix}
,
\begin{pmatrix}
    1 & 2 & 0 \\
    1 & 0 & 2\\
  \end{pmatrix}
,
\begin{pmatrix}
    1 & 1 & 1 \\
    1 & 1 & 1\\
  \end{pmatrix}
, \\
& 
\begin{pmatrix}
    1 & 0 & 2 \\
    1 & 2 & 0\\
  \end{pmatrix}
,
\begin{pmatrix}
    0 & 2 & 1 \\
    2 & 0 & 1\\
  \end{pmatrix}
,
\begin{pmatrix}
    0 & 1 & 2 \\
    2 & 1 & 0\\
  \end{pmatrix}
\}.
\end{align*}

A summary of results and applications of such matrices can be found in \cite{DG95}. In statistical applications these matrices arise as conjugacy tables with fixed margins (alternatively called fixed-margin matrices). Such matrices also play a role in group theory and representation theory. For instance $A^{\lambda}_{\mu}$ is in bijection with double cosets of the symmetric group $\mathfrak{S}_d$ by Young subgroups $\mathfrak{S}_{\lambda}$ and $\mathfrak{S}_{\mu}$. These results and more are summarised in \cite{DG95}.

\section{Fixed-Margin matrix calculator (how to use)}

This github repo contains a single python script to calculate the matrices with margin $(\lambda, \mu)$ for a given $\lambda$, $\mu$. After running the script you will be prompted to enter the row sequence $\lambda$, then the column sequence $\mu$. Enter these as a list of integers separated by commas. If the sum of entries in $\lambda$ and $\mu$ are equal, the console will return the list of all matrices with margin $(\lambda, \mu)$ and tell you how many such matrices there are. Otherwise an error will be raised and you will be prompted to enter the input sequences again. Type $q$ to quit.

\begin{thebibliography}{MOY98}

\bibitem[DG95]{DG95}
Persi Diaconis, Anil Gangolli.
\newblock Rectangular arrays with fixed margins,
\newblock in: {\em Discrete Probability and Algorithms},
\newblock Springer-Verlag, Berlin/New York, 15-41, 1995 

\end{thebibliography}
\end{document}
